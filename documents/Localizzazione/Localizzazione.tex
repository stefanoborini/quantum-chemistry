%% LyX 1.3 created this file.  For more info, see http://www.lyx.org/.
%% Do not edit unless you really know what you are doing.
\documentclass[11pt,a4paper]{article}
\usepackage[T1]{fontenc}
\usepackage[latin1]{inputenc}
\usepackage[italian]{babel}
\usepackage{graphicx}

\makeatletter

%%%%%%%%%%%%%%%%%%%%%%%%%%%%%% LyX specific LaTeX commands.
%% Bold symbol macro for standard LaTeX users
\newcommand{\boldsymbol}[1]{\mbox{\boldmath $#1$}}


%%%%%%%%%%%%%%%%%%%%%%%%%%%%%% User specified LaTeX commands.
\usepackage{verbatim}
\usepackage{babel}
\makeatother
\begin{document}

\title{Uso della catena di localizzazione}

\maketitle

L'uso della catena di localizzazione prevede:

\begin{enumerate}
\item un calcolo CAS per generare lo spazio dei determinanti (file04)
\item un calcolo HF per generare la matrice densita' ad una particella
\item un calcolo utilizzando il programma \textit{schmuds} per generare il
guess di partenza come orbitali localizzati.
\end{enumerate}

Il calcolo CAS per generare lo spazio dei determinanti prevede di effettuare
un calcolo senza simmetria, utilizzando lo switch INTERFACE e DETERMINANTS.
per l'esempio si e' fatto uso della molecola di benzaldeide, secondo questi
file di input
\begin{verbatim}
cat<<EOF>$Project.mol
ATOMBASIS
 $Project
 ano basis
    3    0
      8.     1 ano-1 3 2 1
O1         0.0043716913           -0.0343242164            0.0000000000
       6.     7 ano-1 3 2 1
C1         2.2815322365           -0.0177311117            0.0000000000
C2         3.8476364616            2.3112571884            0.0000000000
C3         6.4783602476            2.0947264271            0.0000000000
C4         7.9938438923            4.2615581716            0.0000000000
C5         6.8676457326            6.6460769596            0.0000000000
C6         4.2269336945            6.8676900157            0.0000000000
C7         2.7192947406            4.7107444955            0.0000000000
       1.     6 ano-1 2 1
H1         7.3440036099            0.2440600759            0.0000000000
H2        10.0279217052            4.0851343463            0.0000000000
H3         8.0311340082            8.3244443945            0.0000000000
H4         3.3599702136            8.7164731110            0.0000000000
H5         0.6849733871            4.8549694770            0.0000000000
H6         3.3349722213           -1.8061464581            0.0000000000
EOF

cat<<EOF>$Project.dal
**DALTON INPUT
!.OPTIMIZE
.RUN WAVE FUNCTION
**INTEGRALS
!.NOTWO
**WAVE FUNCTIONS
.HF
.MP2
.MCSCF
.INTERFACE
*HF INPUT
.HF OCC
28
*AUXILLIARY INPUT
.NOSUPMAT
*CONFIGURATION INPUT
.SYMMETRY
 1
.SPIN MUL
 1
.INACTIVE
 26
.ELECTRONS
 4
.CAS SPACE
 4
*ORBITAL
.NOSUPSYM
!.MOSTART
! NEWORB
*OPTIMIZATION
.DETERMINANTS
.CI PHP MATRIX
300
.MAX MACRO ITERATIONS
100
.MAX CI
150
*CI VECTOR
.PLUS COMBINATIONS
*END OF INPUT
EOF
\end{verbatim}

Vengono creati una serie di file che servono nelle successive fasi. La fase
successiva e' creare un file04 che descriva i determinanti dello spazio cas
generato. per fare questo si fa uso di ijkldali e cipselx, con il seguente
input

\begin{verbatim}
/home/renzo/bin/ijkldali5<<EOF> $OutputDir/ijkldal.out
 &LEGGI
   DIR='$WorkDir/',
   MCORE=400,IOCC=28,MOLAB='CASTOCIP',
   DALTOCIP=T,   &END
EOF
fi

/home/renzo/bin/cipselx <<EOF >$OutputDir/cip.conf.sel.out 2>&1
 &FILES FILE04='$WorkDir/file04',
        FILE03='$WorkDir/file03',
        FILE44='$WorkDir/FILE04',
        FILE25='$WorkDir/FILE25',   &END
 &ICINP ISZ=0,TEST=0.0000,SIGMA=0.10,ZAUTO=T,ZOLD=T,ZBIN=T, &END
EOF
\end{verbatim}

una volta effettuato questo step, e' necessario generare la matrice densita'
ad una particella lanciando un conto HF

\begin{verbatim}
cat<<EOF>$Project.dal
**DALTON INPUT
.RUN WAVE FUNCTION
**INTEGRALS
.NOTWO
**WAVE FUNCTIONS
.HF
.INTERFACE
*HF INPUT
.HF OCC
28
*AUXILLIARY INPUT
.NOSUPMAT
*ORBITAL
.NOSUPSYM
*END OF INPUT
EOF
\end{verbatim}

ed ora si genera il guess localizzato facendo uso del programma schmuds. In
questo file si deve descrivere lo scheletro molecolare legame per legame.

\begin{verbatim}
/home/cele/bin/schmuds <<EOF >$OutputDir/schmuds.out
&smufil prefix='$Project.' /
 &smu nprint=3, orb='$Project.ScfOrb' /
&oao /
 c* 1s(1) pr=1
 o* 1s(1) pr=1
 c* 1s(2) pr=2
 h* 1s(1) pr=2
 o* 1s(2) pr=2
 o* 2px(1) pr=2
 o* 2py(1) pr=2
 o* 2pz(1) pr=2
 c* 2px(1) pr=2
 c* 2py(1) pr=2
 c* 2pz(1) pr=2
fin
 &orb  /
 c o1 1s(1) (1 0)
 c c1 1s(1) (1 0)
 c c2 1s(1) (1 0)
 c c3 1s(1) (1 0)
 c c4 1s(1) (1 0)
 c c5 1s(1) (1 0)
 c c6 1s(1) (1 0)
 c c7 1s(1) (1 0)
 o o1 2py(1) (1 0)
 o c1 1s(2) 2px(1) 2py(1) : c2 1s(2) 2px(1) 2py(1) (1 1)
 o c2 1s(2) 2px(1) 2py(1) : c3 1s(2) 2px(1) 2py(1) (1 1)
 o c3 1s(2) 2px(1) 2py(1) : c4 1s(2) 2px(1) 2py(1) (1 1)
 o c4 1s(2) 2px(1) 2py(1) : c5 1s(2) 2px(1) 2py(1) (1 1)
 o c5 1s(2) 2px(1) 2py(1) : c6 1s(2) 2px(1) 2py(1) (1 1)
 o c6 1s(2) 2px(1) 2py(1) : c7 1s(2) 2px(1) 2py(1) (1 1)
 o c7 1s(2) 2px(1) 2py(1) : c2 1s(2) 2px(1) 2py(1) (1 1)
 o c3 2pz(1) : c4 2pz(1) : c5 2pz(1) : c6 2pz(1) : c7 2pz(1) : c2 2pz(1) (3
3)
 o h1 1s(1) : c3 1s(2) 2px(1) 2py(1) (1 1)
 o h2 1s(1) : c4 1s(2) 2px(1) 2py(1) (1 1)
 o h3 1s(1) : c5 1s(2) 2px(1) 2py(1) (1 1)
 o h4 1s(1) : c6 1s(2) 2px(1) 2py(1) (1 1)
 o h5 1s(1) : c7 1s(2) 2px(1) 2py(1) (1 1)
 o h6 1s(1) : c1 1s(2) 2px(1) 2py(1) (1 1)
 a c1 1s(2) 2px(1) : o1 1s(2) 2px(1) (2 1)
 a c1 2pz(1) : o1 2pz(1) (1 1)
fin




\end{verbatim}









\end{document}
