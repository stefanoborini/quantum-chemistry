\documentclass[a4paper,11pt]{report}
\begin{document}
We start with a suited basis for this type of calculations.

\begin{verbatim}
 &SEWARD  &END
Title
 He2
nopack
Basis set
He...1s.1s. / Inline
          2.   0
    2    2
 2.0 1.0
1.0 0.0
0.0 1.0
 He1         0.0000000000            0.0000000000            0.0
 He2        0.0                     0.0                     4.0
End of basis
End of input
\end{verbatim}

quite crude but effective. The procedure wants a seward run, then an scf run
(with pror 2 100.0 to see all the orbitals) then we need a motra and finally
a molcost with these options

\begin{verbatim}
 &cost prefix='He2.',molcas=54, ycl=T,
 fermi=2,
 /
\end{verbatim}

the ycl=T write a plain text file which contains the bielectronic integrals
in the form

i j k l val

NB: they are (ij|kl), \underline{not} <ij|kl>

and the monoelectonic integrals in the form

i j 0 0 val

please note that these integrals are in molecular basis. 

Given these SCF energies

\begin{verbatim}
   Orb  Occ    Energy  Couls-En    Coefficients

                                   1 1s      1 1s

   1.1   2     -.5473   -2.7908   .848136   .163390

   2.1   0     3.6801    4.8728 -2.333286  2.477686

                                   1 1s      1 1s

   1.5   2     -.5439   -2.7853   .843844   .168207

   2.5   0     3.7096    4.9319 -2.339678  2.481088
\end{verbatim}
 

and these integrals with 4 MO

\begin{verbatim}
    .722797277824      1  1  1  1
    .133337162690      2  1  1  1
    .997625951561E-01  2  1  2  1
    .671586006568      2  2  1  1
    .128286296598      2  2  2  1
    .651038659027      2  2  2  2
    .723420689773      3  3  1  1
    .133785838520      3  3  2  1
    .672166564680      3  3  2  2
    .724046282009      3  3  3  3
    .133196061119      4  3  1  1
    .997697624408E-01  4  3  2  1
    .128256409400      4  3  2  2
    .133644309237      4  3  3  3
    .997783402296E-01  4  3  4  3
    .671313370894      4  4  1  1
    .128056124251      4  4  2  1
    .650736469959      4  4  2  2
    .671893916713      4  4  3  3
    .128023486515      4  4  4  3
    .650441542605      4  4  4  4
    .473418991739      3  1  3  1
    .134550641683      3  2  3  1
    .100308940682      3  2  3  2
    .132430570474      4  1  3  1
    .997680032311E-01  4  1  3  2
    .992367078345E-01  4  1  4  1
    .421737772393      4  2  3  1
    .129117300189      4  2  3  2
    .127193246067      4  2  4  1
    .400736619309      4  2  4  2
   -2.24350979593      1  1  0  0
   -.266358195975      2  1  0  0
    1.19268086390      2  2  0  0
   -2.24138695266      3  3  0  0
   -.267605863011      4  3  0  0
    1.22222567017      4  4  0  0
    1.00000000000      0  0  0  0
\end{verbatim}

and these integrals at 3 MO

\begin{verbatim}
    .651038659027      1  1  1  1
    .672166564680      2  2  1  1
    .724046282009      2  2  2  2
    .128256409400      3  2  1  1
    .133644309237      3  2  2  2
    .997783402296E-01  3  2  3  2
    .650736469959      3  3  1  1
    .671893916713      3  3  2  2
    .128023486515      3  3  3  2
    .650441542605      3  3  3  3
    .100308940682      2  1  2  1
    .129117300189      3  1  2  1
    .400736619309      3  1  3  1
    2.43609028188      1  1  0  0
   -1.26796456486      2  2  0  0
   -.133644311247      3  2  0  0
    2.46561570412      3  3  0  0
   -2.76422231404      0  0  0  0
\end{verbatim}

The value for (1|1) in 3 MO can be evaluated as the
subsequent values from 4MO

$
(2|2)+2*(22|11) - (21|21)
$
$
1.19268086390+2*.671586006568-.997625951561E-01
$

this because the orbitals are renumbered. In the 3 MO calculations the
orbital n. 1 was the 2nd in the 4 MO calculation.

In the 0 0 0 0 term (the nuclear repulsion energy) goes the whole mono and
bielectronic part for the core orbitals. Here the value is (indicating with
(0|0) the nuclear repulsion energy)

$
(0|0)_{3mo} = (0|0)+2*(1|1)+2*(11|11)-(11|11)
$
$
1.0 + 2*(-2.24350979593)+.722797277824
$
this is because, if we write down the expression for the electronic
hamiltonian in spinless formulation

$
H = \sum_{pq} (p|q) E_{pq} + \frac{1}{2} \sum_{pqrs} (pq|rs) (E_{pq}E_{rs} -
\delta_{qr} E_{ps}
$

and remembering that
$
E_{pq} = a^{+}_{p\alpha}a_{q\alpha} + a^{+}_{p\beta}a_{q\beta}
$

the application of the latter on a single determinant gives
\begin{eqnarray}
\left< HF \left| E_{pq} \right| HF \right> & = & \left< HF \left|
a^{+}_{p\alpha}a_{q\alpha} \right| HF \right> + \left< HF \left|
a^{+}_{p\beta}a_{q\beta} \right| HF \right> \\
& = & \delta_{pq} + \delta_{pq} \\
& = & 2 \delta_{pq} 
\end{eqnarray}

and then our hamiltonian operator can be partitioned in sums with indices
that run on the frozen orbitals and indices that run on the other orbitals.
As we can see, the monoelectronic term gives a constant (where both p and q
belongs to a frozen) which will be added to the new nuclear repulsion energy,
two mixed frozen-nonfrozen sums that evaluates always as zero
(because the action of the operator is on different class of orbitals) and
finally a nonfrozen-nonfrozen term that is a real monoelectronic term.
Doing the same for the bielectronic sum, we obtain 16 contributions. Most of
them are zero. The term with the four indices that run on frozen is a
constant and is also merged to the nuclear repulsion energy.
For the others ... (FIXME continue)

To calculate the HF energy, we need to compute, with the 4 MO integrals, the
sum over the occupied orbitals. Please note that the order is 1,3 occupied,
2,4 virtual

$
(1|1)+(11|11)+2*(33|11)-(31|31)
$
$
-2.24350979593+.722797277824+2*.723420689773-.473418991739 = -0.5472901
$
for the scf energy of a virtual orbital, we say

$
(2|2)+2*(22|11)-(21|21)+2*(33|22)-(32|32)
$
$
1.19268086390+2*.671586006568-.997625951561E-01+2*.672166564680-.100308940682
$
Let's try a more difficult calculation. Suppose we have an N2 molecule. This
is the input for molpro

\begin{verbatim}
 ***, N2
 memory,6,m
 int;x,y,z;
 a1,N,0.0,0.0,1.5;
 s,1,roos;3c;
 p,1,roos;1c;
 -;

 hf;
 occ,3,1,1,0,2,0,0,0;
 orbprint,6;

 fci;
 core,2,0,0,0,2,0,0,0;
 dump
\end{verbatim}

The basis set is 3s1p ano basis. The final orbital energies are

\begin{verbatim}
   Orb  Occ    Energy  Couls-En    Coefficients
   
                                   1 1s      1 1s      1 1s      1 2pz
   1.1   2   -15.8083  -42.5719   .999943   .000132   .000180  -.001101
   2.1   2    -1.2403   -9.2615  -.020179   .839029  -.040488  -.201442
   3.1   2     -.5610   -7.5635   .026630   .416299   .064796   .910053
   4.1   0      .4230   -4.1078  -.037609  -.218839   .865665  -.054656
   
                                   1 2px
   1.2   2     -.4908   -7.3221   .918502
   
                                   1 2py
   1.3   2     -.4908   -7.3221   .918502
   
                                   1 1s      1 1s      1 1s      1 2pz
   1.5   2   -15.8077  -42.5719  1.000125   .001044   .001135  -.001405
   2.5   2     -.9255   -8.3026   .008110  1.033540  -.021625   .197562
   3.5   0      .2076   -7.1786  -.049957  -.614440   .118351  1.192354
   4.5   0      .7374   -4.0818   .122885  1.075876  1.709992  -.774165

                                   1 2px
   1.6   0     -.0357   -7.1433  1.107922

                                   1 2py
   1.7   0     -.0357   -7.1433  1.107922
\end{verbatim}
 

let's try to evaluate the energy for the first orbital. we have

\begin{eqnarray}
(1|1)&+&(11|11) \\
  &+&2*(22|11)-(21|21) \\
  &+&2*(33|11)-(31|31) \\
  &+&2*(55|11)-(51|51) \\
  &+&2*(66|11)-(61|61) \\
  &+&2*(77|11)-(71|71) \\
  &+&2*(88|11)-(81|81)
\end{eqnarray}


the order for the orbitals is:
\begin{verbatim}
1.1 -> 1    2.1 -> 2    3.1 -> 3    4.1 -> 4   1.2 -> 5
1.3 -> 6    1.5 -> 7    2.5 -> 8    3.5 -> 9
4.5 -> 10   1.6 -> 11   1.7 -> 12
\end{verbatim}

occupied orbitals are 1 2 3 5 6 7 8 

\begin{verbatim}
    2.22985979346      1  1  1  1
    .300875029048E-01  2  1  2  1
    .670739004262      2  2  1  1
    .146710493852E-01  3  1  3  1 
    .619586217862      3  3  1  1
    .596164398969      5  5  1  1
    .122817637216E-01  5  1  5  1
    .596164398969      6  6  1  1  
    .122817637216E-01  6  1  6  1
    2.23008190655      7  7  1  1 
    1.89671151159      7  1  7  1
    .650259273947      8  8  1  1  
    .344929011568E-01  8  1  8  1
   -26.7636283326      1  1  0  0
    16.3333333333      0  0  0  0
\end{verbatim}

\begin{eqnarray}
-26.7636283326 & + & 2.22985979346 \\
 &+&2*.670739004262-.300875029048E-01 \\
 &+&2*.619586217862-.146710493852E-01 \\
 &+&2*.596164398969-.122817637216E-01 \\
 &+&2*.596164398969-.122817637216E-01 \\
 &+&2*2.23008190655-1.89671151159 \\
 &+&2*.650259273947-.344929011568E-01 \\
&=& -15.808305
\end{eqnarray}

so far so good. Let's try our real game.
first of all: the renumeration.

\begin{verbatim}
 occ,3,1,1,0,2,0,0,0;
core,2,0,0,0,2,0,0,0;
\end{verbatim}

so it means that the orbitals marked as 1,2,7,8 are no more in play.
For this reason, the remaining active orbitals should be

\begin{verbatim}
3  ->  1 (in mo restricted)
4  ->  2 ( " )
5  ->  3 ( " )
6  ->  4 ( " )
9  ->  5 ( " )
10 ->  6 ( " )
11 ->  7 ( " )
12 ->  8 ( " )
\end{verbatim}

(check ORBSYM=1,1,2,3,5,5,6,7 from output... it is coherent. ok)

so the value for (1|1) in the small MO evaluation should be

(3|3)+2*(33|11) - (31|31)
     +2*(33|22) - (32|32)
     +2*(33|77) - (37|37)
     +2*(33|88) - (38|38)

which should evaluate as 

 -2.79916184443      1  1  0  0



 -7.00255791386 + 2*.443886537843 - .146710493852E-01
                + 2*.488633828559 - .693573270881E-01

                + 2*.619590157014 - .143434456282E-01
                + 2*.490734169087 - .135320853512



%%%%%%%%%%%%%%%%%%%%%%%%%%%%%%%%%%%%%%%

The nuclear repulsion energy contribution for the 0 frozen case evaluates
as 16.3333333333, whereas the same value for 4 frozen is -98.8640342995.
This should be

$
(0|0) restricted = (0|0) + \sum_{i=1,2,7,8} 2*(i|i) + \sum_{i,j=1,2,7,8} 2*(ii|jj) - (ij|ij)
$

%%%%%%%%%%%%%%%%%%%%%%%%%%%%%%%%%%%%%%%

\section{Appendix A}

Demonstration of the sums can be given here.
Being this the hamiltonian, written in spatial orbital formulation

$
H = \sum_{pq} (p|q) E_{pq} + \frac{1}{2} \sum_{pqrs} (pq|rs) (E_{pq}E_{rs} -
\delta_{qr} E_{ps}
$

and remembering that
$
E_{pq} = a^{+}_{p\alpha}a_{q\alpha} + a^{+}_{p\beta}a_{q\beta}
$

and so the application of the latter on a single determinant gives
\begin{eqnarray}
\left< HF \left| E_{pq} \right| HF \right> & = & \left< HF \left|
a^{+}_{p\alpha}a_{q\alpha} \right| HF \right> + \left< HF \left|
a^{+}_{p\beta}a_{q\beta} \right| HF \right> \\
& = & \delta_{pq} + \delta_{pq} \\
& = & 2 \delta_{pq} 
\end{eqnarray}

and then our hamiltonian operator can be partitioned in sums with indices
that run on the frozen orbitals and indices that run on the other orbitals.
As we can see, the monoelectronic term gives a constant (where both p and q
belongs to a frozen) which will be added to the new nuclear repulsion energy,
two mixed frozen-nonfrozen sums that evaluates always as zero
(because the action of the operator is on different class of orbitals) and
finally a nonfrozen-nonfrozen term that is a real monoelectronic term.
Doing the same for the bielectronic sum, we obtain 16 contributions. Most of
them are zero. The term with the four indices that run on frozen is a
constant and is also merged to the nuclear repulsion energy.
For the others ... (FIXME continue)

\end{document}
