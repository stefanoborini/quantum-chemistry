\documentclass[a4paper]{article}
\begin{document}

strategy:

for each of the molecules formaldehyde, acetaldehyde, acetone, we need to compute

1) vertical energies for excited states n->pi* and pi->pi* singlet and triplet
2) adiabatic energies (need geometry optimization... need to solve)

Strategy for the active space is here depicted: for each of the excited states
we compute, with the localization strategy, an average between the ground and
the excited state. The active space must be the minimal to achieve a good description
for the excitation. For this reason, we target at ny,pi* localized space and pi,pi*
localized space for our excitations.

We need also to evaluate the transitions at the DDCI and CAS+SD level.

Problem n.1: the localized optimization procedure converges at a minimum that is not
the CASSCF minimum, due to the uncontracted approach for the CAS+S strategy. Also,
to submit our orbitals to nevpt, we need a canonization. We will use this procedure:

1) Localization chain to obtain localized optimized orbitals (they aren't the CASSCF
   orbitals, but they are really near)

2) RASSCF calculation (with molcas) using this orbital set to converge to the
   CASSCF orbitals plus canonization. This should indeed delocalize the orbitals,
   but it shouldn't corrupt the nature of the active space (we hope)

3) NEVPT chain with CASSCF+canonize orbitals

Problem n. 2: localization can optimize for an average of states with
different symmetry, but molcas does not. This lead to the problem that doing
an average optimization between the ground state and the n->pi* state, the
obtained energy (and orbitals) are not optimized for the ground nor for the
excited. The subsequent molcas treatment (needed as long as we need canonized
casscf orbitals) converges at the ground \underline{or} at the excited, which leads to
heavy delocalization and corruption of the cas space nature. To target this
point there are three strategies:

1) embed the canonization in the optimization procedure in noscf. not
feasible.

2) skip the average optimization and locally optimize for each state.
This strategy forces to an SCF for the ground state, since there's no
need for static correlation, and recover the dynamic correlation with a simple
MP2 treatment. 

3) completely remove the symmetry.
In this case, after obtaining the optimized orbitals for the average for the
ground and excited state, we need to converge at the average casscf solution
with molcas (now the states are on the same symmetry, so it's no longer a
problem), then extract each state with a rasci+canonize and feed these
canonized orbitals into the nevpt2 chain.

Also, check the geometries! our geometries were optimized at CAS(6/5) level.
bby assumption, we can trust them as a fixed point for comparison against
lara's data.


1(n->pi*)
---------
n and pi* in CAS

* CAS2/2-loc+NEVPT2 average\_CAS nosymm

ground
CAS -113.89828957
SC  -114.26953357
PC  -114.26954271


excited

CAS -113.76781143 (3.55)
SC  -114.12343397 (3.98)
PC  -114.12343900 (3.98)

* CAS2/2-deloc+NEVPT2 average\_CAS nosymm

ground
CAS  -113.89828954
SC   -114.26953357
PC   -114.26954271

excited
CAS  -113.76781143 (3.55)
SC   -114.12343391 (3.98)
PC   -114.12343894 (3.98)


1(pi->pi*)
----------
pi and pi* in cas

CAS2/2-loc+NEVPT2 average\_CAS nosymm

there are problems in convergence, due to the fact that the n->pi* state lays
under the pi->pi* one. For this reason, the optimization procedure tries to
include this state, and at the end of the optimization there are pi and pi* as
active space, but the small coefficients on the n->pi* one are dramatic for
the subsequent casscf procedure. For this reason, a change of strategy is
needed. We set the active space to 4 electrons / 3 orbitals, including n,pi
and pi* orbitals. The procedure is the same.


CAS3/4-loc+NEVPT2 average\_CAS nosymm

           ground          n->pi*                 pi->pi*

CAS   -113.93011520    -113.78485967 (3.95)   -113.51725460 (11.23)
SC    -114.25832978    -114.10849912 (4.08)   -113.91406971 (9.37)
PC    -114.25922520    -114.11001484 (4.06)   -113.92883896 (8.99)


Doing the same calculation without localization we obtain the same values, so
we cannot address the problem for the wrong evaluation of the pi->pi* to the
localization procedure.

Also, a CAS+SD evaluation has been performed:

-114.25206281 -114.10725960 (3.94) -113.88372567 (10.02)



The same calculations on acetone gives

CAS   -192.04587952    -191.88416488 (4.40)   -191.64643339 (10.87)
SC    -192.68869092    -192.52241498 (4.52)   -192.35751421 (9.01)
PC    -192.68966691    -192.52404599 (4.51)   -192.37131301 (8.66)


why the discrepancy ? maybe the NEVPT results are heavily affected by the
presence of the sigma and sigma* orbitals?
since we cannot build a localized space with 6/5 (due to problem in
convergence with orbitals that are nearly empty) nor hold a 6/5 space without
localization and symmetry (since we need to do average cas, and we need to
remove symmetry to do the average with molcas), we try another strategy and
reimpose symmetry, then we calculate the energy for the pi/pi* state with the
sigma/sigma* and without. An average cas between the ground and the pi->pi*
state is performed.

 
with sigma, these energies are obtained

ground   -113.96386577           -114.24543946           -114.24875572
pi->pi*  -113.57819232 (10.49)   -113.87946329 (9.95)    -113.89230612 (9.70)

without sigma

ground   -113.92447070           -114.25828560           -114.25904939
pi->pi*  -113.53112610 (10.70)   -113.90099214 (9.72)    -113.91291415 (9.41)


Weighting differently the ground and pi->pi* we obtain

(4.0 1.0 4.0)

CAS  -113.92696969      -113.77210112 (4.21)     -113.52753067 (10.86)
SC   -114.25865170      -114.11161686 (4.00)     -113.90448914 (9.63)
PC   -114.25945931      -114.11376198 (3.96)     -113.91757130 (9.30)

(6.0 1.0 6.0)

CAS  -113.92628054      -113.76945857 (4.27)     -113.52876723 (10.81)
SC   -114.25854600      -114.11207177 (3.98)     -113.90327035 (9.66)
PC   -114.25934502      -114.11434733 (3.94)     -113.91608013 (9.34)

(10 1 10)
 
CAS  -113.92563500      -113.76697632 (4.32)     -113.52974165 (10.77)
SC   -114.25841376      -114.11244304 (3.97)     -113.90227226 (9.69)
PC   -114.25920633      -114.11484368 (3.93)     -113.91483356 (9.37)
 
(20 1 20)

CAS  -113.92508294      -113.76483878 (4.36)     -113.53045290 (10.73)
SC   -114.25828242      -114.11273840 (3.96)     -113.90151306 (9.71)
PC   -114.25907005      -114.11524762 (3.91)     -113.91386163 (9.39)



































\end{document}
