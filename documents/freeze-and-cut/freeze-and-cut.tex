\documentclass[11pt,a4paper]{article}
\usepackage[T1]{fontenc}
\usepackage[latin1]{inputenc}
\usepackage[italian]{babel}
\usepackage{graphicx}

\makeatletter

%%%%%%%%%%%%%%%%%%%%%%%%%%%%%% LyX specific LaTeX commands.
%% Bold symbol macro for standard LaTeX users
\newcommand{\boldsymbol}[1]{\mbox{\boldmath $#1$}}


%%%%%%%%%%%%%%%%%%%%%%%%%%%%%% User specified LaTeX commands.
\usepackage{verbatim}
\usepackage{babel}
\makeatother
\begin{document}



\section{Freeze and cut}

The freeze and cut technique allows to perform a localized CAS optimization on
a restricted set of atoms.
The calculation takes place in three steps:

\begin{enumerate}
\item a large calculation on the molecule as a whole
\item a small calculation on a reduced subset of the molecule
\item the optimization procedure
\end{enumerate}

The first step is needed to generate a localized set of orbitals expressed
on fewer atoms and the one electron integrals and nuclear repulsion energy
modified by the freeze (freezing orbitals effectively turns one electron
integrals concerning these orbitals into costants, which fall into the
nuclear energy, and two electron integrals into one electron ones).

The second step generates the two electron integrals for the smaller system
obtained by removing the atoms (cut) and the appropriate overlap.

The third step performs the optimization. Since this step acts into the
reduced set of atomic basis functions obtained by the cut, this optimization
cannot involve atoms that were excluded by the previous step.


Let's see in more detail each step

\subsection{first step - large system}

The first step needs to perform an SCF evaluation on the complete system,
thus a seward must be performed and then the SCF evaluation.

\begin{verbatim}
 &SEWARD  &END
Title
 tridequenal
square
Basis set
C.ano-l...2s1p.
 C1  0.00000000     0.00000000     0.00000000 Angstrom
 C2 -1.25573684     0.00000000    -0.72500000 Angstrom
 C3 -2.42487113     0.00000000    -0.05000000 Angstrom
 C4 -3.68060797     0.00000000    -0.77500000 Angstrom
 C5 -4.84974226     0.00000000    -0.10000000 Angstrom
 C6 -6.10547910     0.00000000    -0.82500000 Angstrom
 C7 -7.27461339     0.00000000    -0.15000000 Angstrom
End of basis
Basis set
O.ano-l...2s1p.
 O1   0.0000000    0.000000     1.220000  Angstrom
End of basis
Basis set
H.ano-l...1s.
 H1    0.95262794    0.000000    -0.550000 Angstrom
 H2   -1.25573684    0.00000000    -1.82500000 Angstrom
 H3   -2.42487113    0.00000000     1.05000000 Angstrom
 H4   -3.68060797    0.00000000    -1.87500000 Angstrom
 H5   -4.84974226    0.00000000     1.00000000 Angstrom
 H6   -6.10547910    0.00000000    -1.92500000 Angstrom
 H7   -7.27461339    0.00000000     0.95000000 Angstrom
End of basis
Basis set
C.ano-l...2s1p.
 C8 -8.53035023     0.00000000    -0.87500000 Angstrom
 C9 -9.69948452     0.00000000    -0.20000000 Angstrom
 C10 -10.95522136     0.00000000    -0.92500000 Angstrom
 C11 -12.12435565     0.00000000    -0.25000000 Angstrom
 C12 -13.38009249     0.00000000    -0.97500000 Angstrom
 C13 -14.54922678     0.00000000    -0.30000000 Angstrom
End of basis
Basis set
H.ano-l...1s.
 H8   -8.53035023    0.00000000    -1.97500000 Angstrom
 H9  -9.69948452     0.00000000     0.90000000 Angstrom
 H10  -10.95522136     0.00000000    -2.02500000 Angstrom
 H11  -12.12435565     0.00000000     0.85000000 Angstrom
 H12  -13.38009249     0.00000000    -2.07500000 Angstrom
 H131  -14.54922678     0.00000000     0.80000000 Angstrom
 H132  -15.50185473     0.00000000    -0.85000000 Angstrom
End of basis
End of input
\end{verbatim}

\begin{verbatim}
 &SCF &END
Title
 tridequenal
occupied
50
end of input
\end{verbatim}

Using molcas, the produced files must be passed to our chain, so a molcost
transformation must be performed

\begin{verbatim}
 &cost prefix='tridequenal.',molcas=54,
 fermi= 50,
 /
\end{verbatim}

The schmuds chain is now invoked. schmudorb provides the localized orbitals,
proj\_scf performs the projection of these orbitals on the SCF ones to obtain
localized SCF orbitals and schmudort performs a hierarchical class based
orthogonalization.


\begin{verbatim}
 &smufil prefix='tridequenal.',progr='MOLCAS' /
 &smu nprint=1
 orb='tridequenal.ScfOrb' /

 &oao /
C* 1s(1) pr=1
C* 1s(2) pr=2
C* 2px(1) pr=2
C* 2py(1) pr=2
C* 2pz(1) pr=2
O* 1s(1) pr=1
O* 1s(2) pr=2
O* 2px(1) pr=2
O* 2py(1) pr=2
O* 2pz(1) pr=2
H* 1s(1) pr=2
fin

 &orb /
G_O1 O1 1s(1) (1 0)
G_C1 C1 1s(1) (1 0)
G_C2 C2 1s(1) (1 0)
G_C3 C3 1s(1) (1 0)
G_C4 C4 1s(1) (1 0)
G_C5 C5 1s(1) (1 0)
G_C6 C6 1s(1) (1 0)
G_C7 C7 1s(1) (1 0)
G_C8 C8 1s(1) (1 0)
G_C9 C9 1s(1) (1 0)
G_C10 C10 1s(1) (1 0)
G_C11 C11 1s(1) (1 0)
G_C12 C12 1s(1) (1 0)
G_C13 C13 1s(1) (1 0)
G_C3C4 C3 1s(2) 2p(1) : C4 1s(2) 2p(1) (1 1)
G_C4C5 C4 1s(2) 2p{x,z}(1) : C5 1s(2) 2p{x,z}(1) (1 1)
G_C5C6 C5 1s(2) 2p(1) : C6 1s(2) 2p(1) (1 1)
G_C6C7 C6 1s(2) 2p{x,z}(1) : C7 1s(2) 2p{x,z}(1) (1 1)
G_C7C8 C7 1s(2) 2p(1) : C8 1s(2) 2p(1) (1 1)
G_C8C9 C8 1s(2) 2p{x,z}(1) : C9 1s(2) 2p{x,z}(1) (1 1)
G_C9C10 C9 1s(2) 2p(1) : C10 1s(2) 2p(1) (1 1)
G_C10C11 C10 1s(2) 2p{x,z}(1) : C11 1s(2) 2p{x,z}(1) (1 1)
G_C11C12 C11 1s(2) 2p(1) : C12 1s(2) 2p(1) (1 1)
G_C12C13 C12 1s(2) 2p{x,z}(1) : C13 1s(2) 2p{x,z}(1) (1 1)
G_C4C5 C4 2py(1) : C5 2py(1) (1 1)
G_C6C7 C6 2py(1) : C7 2py(1) (1 1)
G_C8C9 C8 2py(1) : C9 2py(1) (1 1)
G_C10C11 C10 2py(1) : C11 2py(1) (1 1)
G_C12C13 C12 2py(1) : C13 2py(1) (1 1)
G_C3H3 C3 1s(2) 2p(1) : H3 1s(1) (1 1)
G_C4H4 C4 1s(2) 2p(1) : H4 1s(1) (1 1)
G_C5H5 C5 1s(2) 2p(1) : H5 1s(1) (1 1)
G_C6H6 C6 1s(2) 2p(1) : H6 1s(1) (1 1)
G_C7H7 C7 1s(2) 2p(1) : H7 1s(1) (1 1)
G_C8H8 C8 1s(2) 2p(1) : H8 1s(1) (1 1)
G_C9H9 C9 1s(2) 2p(1) : H9 1s(1) (1 1)
G_C10H10 C10 1s(2) 2p(1) : H10 1s(1) (1 1)
G_C11H11 C11 1s(2) 2p(1) : H11 1s(1) (1 1)
G_C12H12 C12 1s(2) 2p(1) : H12 1s(1) (1 1)
G_C13H131 C13 1s(2) 2p(1) : H131 1s(1) (1 1)
G_C13H132 C13 1s(2) 2p(1) : H132 1s(1) (1 1)
O_C1C2 C1 1s(2) 2p(1) : C2 1s(2) 2p(1) (1 1)
O_C2C3 C2 1s(2) 2p{x,z}(1) : C3 1s(2) 2p{x,z}(1) (1 1)
A_C2C3 C2 2py(1) : C3 2py(1) (1 1)
O_O1 O1 1s(2) 2p(1) (2 0) Ref 1
O_O1C1 C1 1s(2) 2p{x,z}(1) : O1 1s(2) 2p{x,z}(1) (3 1) Proj 1
A_O1C1 C1 2py(1) : O1 2py(1) (1 1)
O_C1H1 C1 1s(2) 2p(1) : H1 1s(1) (1 1)
O_C2H2 C2 1s(2) 2p(1) : H2 1s(1) (1 1)
fin

 &ort /
 &pscf prefix='tridequenal.'  /

\end{verbatim}


as we can see from the input we are going to freeze a big chunk of the
molecule. remember that proj\_scf produces a file called NONORLOC\_scf which
must be moved to NONORLOC before running the orthogonalization schmudort.

Now we run troncat, the central point for the cut part of the technique.

\begin{verbatim}
 &TRONCAT prefix='tridequenal.'
 nprint_hieror=-1,
 keepl='C1C2','C2C3','O1','O1C1','C1H1','C2H2'
 keep='O1','C1','C2','C3'
 'H1','H2' /
\end{verbatim}

As we can see, the list of bonds to keep must be specified (everything else
is considered and marked frozen) and the list of atoms which will
characterize the reduced system.

troncat procudes a lot of interesting data. from the output

\begin{verbatim}
 Kept Atomic Orbitals
C1  1s   | C1  1s   | C1  2px  | C1  2py  | 
C1  2pz  | C2  1s   | C2  1s   | C2  2px  |
C2  2py  | C2  2pz  | C3  1s   | C3  1s   |
C3  2px  | C3  2py  | C3  2pz  | O1  1s   |
O1  1s   | O1  2px  | O1  2py  | O1  2pz  |
H1  1s   | H2  1s   | 
 Eliminated Atomic Orbitals
C4  1s   | C4  1s   | C4  2px  | C4  2py  |
C4  2pz  | C5  1s   | C5  1s   | C5  2px  |
C5  2py  | C5  2pz  | C6  1s   | C6  1s   |
C6  2px  | C6  2py  | C6  2pz  | C7  1s   |
C7  1s   | C7  2px  | C7  2py  | C7  2pz  |
H3  1s   | H4  1s   | H5  1s   | H6  1s   |
H7  1s   | C8  1s   | C8  1s   | C8  2px  |
C8  2py  | C8  2pz  | C9  1s   | C9  1s   |
C9  2px  | C9  2py  | C9  2pz  | C10 1s   |
C10 1s   | C10 2px  | C10 2py  | C10 2pz  |
C11 1s   | C11 1s   | C11 2px  | C11 2py  |
C11 2pz  | C12 1s   | C12 1s   | C12 2px  |
C12 2py  | C12 2pz  | C13 1s   | C13 1s   |
C13 2px  | C13 2py  | C13 2pz  | H8  1s   |
H9  1s   | H10 1s   | H11 1s   | H12 1s   |
H1311s   | H1321s   | 


....

 Overlap with non-truncated Orb          Norm
      O_C1C2   42        1.000815        1.002692
      O_C2C3   43        1.025093        1.031652
        O_O1   44        0.999967        1.000058
        O_O1   45        1.000352        1.000458
      O_O1C1   46        1.000463        1.000624
      O_C1H1   47        1.000205        1.000488
      O_C2H2   48        1.003978        1.007743
      A_C2C3   49        0.986023        0.994496
      A_O1C1   50        0.998465        0.999312
      A_C2C3   51        0.967466        1.015485
      A_O1C1   52        0.995299        0.998864
      O_C1C2   53        1.000765        1.006009
      O_C2C3   54        1.005900        1.034265
      O_O1C1   55        1.000776        1.001562
      O_C1H1   56        1.000241        1.001147
      O_C2H2   57        0.999536        1.002313

\end{verbatim}

It's important to note that the norm can be greater than 1 because we are
working in a nonorthogonal basis due to the cut.

\begin{verbatim}
 ngo,nngo,nngv,ngv    41  9  7  27
         16  orbitales moleculaires locales gardees
          22  orbitales atomiques gardees
\end{verbatim}

Here we can note that we froze 41 core orbitals and 27 virtual orbitals. Our
reduced system will have 16 orbitals, expressed on 22 atom basis. 
For these orbitals, the expression on atomic basis different from the kept
ones is set to zero, and the submatrix is extracted. We obtain two
matrices: the first one (stored in TRONCORBG) has the full dimensionality
but coefficient set to zero, and the other one (stored in TRONCORBP) has the
reduced dimensionality.
For compatiblity reasons with molcas, which is used to perform AO/MO
transformation, we cannot feed a rectangular matrix to motra. For this
reason, the 16x22 matrix is squared with other 6 orbitals marked as G\_BID
with no coefficients. The strategy is to trick motra to ignore these fake
orbitals passing an appropriate delete parameter.
Troncat also produces other important files: a .Mono file for the reduced
system, named TRONCP\_MONO, which contains the AO overlap for the reduced
basis set; two input files for the subsequent hieriarchical
orthonormalization, hierinp and hiering, and a MASK file which is used to
reinsert the small matrix 16x22 into the large one. This MASK reproduces the
geometry of the large matrix, and has a 1 for each element which belongs to
the small one, or a 0 elsewhere. 

Then we need to run hieror for both the systems. Important to note is that
hieror expect the orbitals in INPORB and the .Mono in MONO, so an
appropriate copy must be performed for TRONCORBP (along with TRONCP\_MONO)
and TRONCORBG (with the original .Mono from the first molcost). hieror
produces a ORTORB file which is renamed to ORTORBP or ORTORBG for small and
large.

The first part is finished, except for a final step: we need to create the
integrals in MO basis from the new orbitals on the complete basis

\begin{verbatim}
 &MOTRA &END
Title
 tridequenal
LumOrb
Frozen
41
Delete
27
Onel
End of input
\end{verbatim}

and the obvious conversion into the cost format

\begin{verbatim}
 &cost prefix='tridequenal.',molcas=54,
 fermi=9, frozen=41, oneonly=T,
 /
\end{verbatim}

obviously we convert only the one-electron integrals because we are
interested only in this object.


\subsection{Second step - small system}

for the small system we need to perform the same step as before,
transforming the AO integrals from the reduced system in MO basis.
We perform a seward on our reduced system

\begin{verbatim}
 &SEWARD  &END
Title
 tridequenal
square
Basis set
C.ano-l...2s1p.
 C1  0.00000000     0.00000000     0.00000000 Angstrom
 C2 -1.25573684     0.00000000    -0.72500000 Angstrom
 C3 -2.42487113     0.00000000    -0.05000000 Angstrom
End of basis
Basis set
O.ano-l...2s1p.
 O1   0.0000000    0.000000     1.220000  Angstrom
End of basis
Basis set
H.ano-l...1s.
 H1    0.95262794    0.000000    -0.550000 Angstrom
 H2   -1.25573684    0.00000000    -1.82500000 Angstrom
End of basis
End of input
\end{verbatim}

and then we transform the integrals with the orbitals from ORTORBP, not
forgetting that we need to drop the fake orbitals to have a square matrix
(to satisfy motra).

\begin{verbatim}
 &MOTRA &END
Title
 Tridequenal
LumOrb
delete
6
End of input
\end{verbatim}

and then we perform the usual molcost step

\begin{verbatim}
 &cost prefix='tridequenal.', molcas=54,
 fermi=9,
 /
\end{verbatim}

And the second step is finished.


\subsection{third step - optimization}

The third step uses the files provided by the first two steps in order to
perform the optimization.

The troncoptimizer script is a bash script that needs files in appropriate
subdirectories of a base directory (which must be changed inside the script)
inside the scratch

\begin{verbatim}
basedir/start
basedir/seward-large
basedir/seward-small
\end{verbatim}

seward-large is a directory which must contain the seward binary files obtained from the
molcas run on the complete system. seward-small contains the seward  binary files
from the reduced system.

start must contain
\begin{itemize}
\item ORTORBP, ORTORBG and MASK
\item the .Mono and Info files obtained at the end of the first step,
renamed as .Mono.large and .Info.large
\item the .Mono and Info files obtained at the end of the second step,
renamed as .Mono.small and .Info.small
\item the .ijcl file obtained at the end of the second step.
\end{itemize}

with this setup, the troncoptimizer script is ready to run. This script
performs a high number of steps and iterates to improve the orbitals at each
iteration. The core of the behavior is to improve the ORTORBP small
matrix. At each iteration, the improved matrix is reinserted inside the
large ORTORBG using the MASK, and a new integral transformation must be
performed both on the large system (for one el) and small system (for two
el). The procedure automatically obtains a new set of files like the ones
provided by the user in the start directory, and a new iteration begins. The
stop criterium is on the variation of the energy. If the energy is constant
for two iterations, the procedure stops.

The input files needed for this step are
\begin{verbatim}
SCH_FERMI
tridequenal.casdi.in
tridequenal.molcost-bi.in
tridequenal.motra-bi.in
tridequenal.casdi-cas.in
tridequenal.locnats.in
tridequenal.molcost-mono.in
tridequenal.motra-mono.in
\end{verbatim}

Each file will be discussed along the following explanation.

The optimization procedure works as follows: first the .Mono from large and
small are mixed together to obtain a new .Mono. This mix is required since
we need the one electron integrals from the large calculation but the
AO overlap which is contained inside this file is relative to a large basis
set. We need the overlap on the reduced basis set to perform correctly, so
the utility program mixmono perform this mixing.
Also, we need to mix the .Info, because the nuclear energy evaluated for the
small system is fake (we need the nuclear energy corrected by the frozen
contribution which is held inside the .Info.large). This mixing is performed
by a simple internal function in troncoptimizer.

Now everything we need to perform the SuperCI evaluation is available: we run
casdet/casdi to perform a CAS+S evaluation, using this tridequenal.casdi.in

\begin{verbatim}
 &cdfil prefix='tridequenal.',
  det1='DET001',  /
 &cd gener='CAS+S',noac=4,numac=8,9,10,11,
 nelac=4,ms2=0,is0=1,  /
 &cdifil prefix='tridequenal.',
 det1='DET001', davec='DAVEC001',
 dens1='DENS001',  /
 &dav nvec=1,syspin='+',iprec=6, /
\end{verbatim}

in this case, a CAS 4/4 is defined and a CAS+S evaluation is performed,
providing the energy. Since it's a CAS+S optimization, the CAS optimized
wavefunction is obtained when there's no interaction with the singles
excitations of the multireference wavefunction (extended brillouin theorem
EBT).
When this is verified, no variation of the energy occurs adding the single
excitations. The convergence criterium is to check this energy when it
becomes constant against the iterations.
The procedure creates a file named tridequenal.DAVEC001, which the
troncoptimized moves to ESS001 to feed the subsequent locnats. Also, the
ORTORBP file is copied to INPORB.
locnats is now able to perform the localized optimization

\begin{verbatim}
&locn
  prefix='tridequenal.'
  ngel=0,
  ndel=6,
  fermi=9,
  nmat=1,
  seuil= 0.000100000000000,
  densmat='DENS001',
  netat=1,
  metat=1,
  coef=1.0d0,
  orbv='INPORB',
  orbn='LOCORB2',
  nprint=5,
  yverif=T,
 /
\end{verbatim}

also using the provided SCH\_FERMI

\begin{verbatim}
&ferm fermi= 9, /
\end{verbatim}

please note that the fermi level is always specified relative to the reduced
non-frozen system of orbitals, thus the 9.
the locnats input is relative to ORTORBP, so the number of frozen orbitals
is zero (because the small matrix has no "core" removed orbitals) and the
number of deleted orbitals (the "virtual" removed orbitals) is the number of
fake orbitals in the ORTORBP matrix.

The result is provided by locnats in the file LOCORB2.

The troncoptimizer now has what it needs: an improved set of orbitals on the
small system. At this point, to recover the starting position it needs to
retransform integrals from AO to MO with the new orbitals. To recreate the
two electron integrals, it needs only the seward-small and the LOCORB2
(moved to INPORB), and run motra and then molcost.

\begin{verbatim}
 &MOTRA &END
Title
 tridequenal
LumOrb
Delete
6
End of input
\end{verbatim}

\begin{verbatim}
 &cost prefix='tridequenal.',molcas=54,
 fermi= 9,
 /
\end{verbatim}

To recreate the one electron .Mono and .Info for the large system we need a
previous step: reinsertion of the small matrix into the old large matrix,
replacing only the elements interested in optimization, and not touching the
remaining part. The program expandat performs this service, taking the
ORTORBP, ORTORBG and MASK and creating a file MIXORB. No user intervention
is needed. The troncoptimizer now copy MIXORB to INPORB and motra/molcost is
performed to obtain the .Mono and .Info

here is the motra-mono.in

\begin{verbatim}
 &MOTRA &END
Title
 tridequenal
LumOrb
Frozen
41
Delete
27
Onel
print
10
End of input
\end{verbatim}

and the molcost-mono.in

\begin{verbatim}
 &cost prefix='tridequenal.',molcas=54,
 fermi=9, frozen=41, oneonly=T,
 /
\end{verbatim}

All the staring files have been created. They can be collected and a new
iteration can begin. When the EBT is verified, the energy given by casdi
does not change, the last iteration collects all files and start the
postprocessing procedure, which is a standard CAS evaluation

\begin{verbatim}
 &cdfil prefix='tridequenal.'  /
 &cd gener='CAS',noac=4,numac=8,9,10,11,
 nelac=4,ms2=0,is0=1,  /
 &cdifil prefix='tridequenal.'  /
 &dav nvec=1, syspin='+', iprec=6 /
\end{verbatim}

As a reference, this is the output of the energies for the represented
example

\begin{verbatim}
Iteration 1: -573.88893768
Iteration 2: -573.88905028
Iteration 3: -573.88904383
Iteration 4: -573.88904334
Iteration 5: -573.88904326
Iteration 6: -573.88904326
\end{verbatim}


\end{document}
