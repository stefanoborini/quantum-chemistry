\documentclass[11pt,a4paper]{article}
\usepackage[T1]{fontenc}
\usepackage[latin1]{inputenc}
\usepackage[italian]{babel}
\usepackage{graphicx}

\makeatletter

%%%%%%%%%%%%%%%%%%%%%%%%%%%%%% LyX specific LaTeX commands.
%% Bold symbol macro for standard LaTeX users
\newcommand{\boldsymbol}[1]{\mbox{\boldmath $#1$}}


%%%%%%%%%%%%%%%%%%%%%%%%%%%%%% User specified LaTeX commands.
\usepackage{verbatim}
\usepackage{babel}
\makeatother
\begin{document}



\section{How to use localization}

The localization chain is bound to molcas through the molcost interface.
molcost takes the output from molcas and convert it in a file format useful for
the localization chain. An analogous program could be implemented for
dalton, named dalcost.

The normal procedure is given by

\begin{itemize}
\item seward 
\item scf
\item molcost
\item schmudorb
\item proj\_scf
\item schmudort
\item noscf
\end{itemize}

The seward input defines everything as usual. Only a couple of things should be
kept in mind. first: you need ANO basis, second: the label for each atom is
important for the subsequent steps, so keep them different one from each other,
and coherent to recover them in the next steps.

The SCF input is quite simple. You need to specify the occupation for each
symmetry. Nothing complicated.

Then it cames molcost. The input is also really simple. You need to specify
only the fermi level

\begin{verbatim}
 &cost prefix='pentenal.' , molcas=54,
 fermi=19,3,
 /
\end{verbatim}

Probably molcost will complain about a missing file, but it's not
important. At the first run, you need only to create the .Info file that
will be used in subsequent steps. Molcost generates 3 files: .Info is a
namelist that holds various informations about the calculation done, .Mono
holds the AO overlap matrix and the monoelectronic MO integrals, .ijcl holds
the MO bielectronic integrals.

After this you need to run schmudorb. This code generates the localized guess.
This is for the case of penthenal

\begin{verbatim}
 &smufil prefix='pentenal.',progr='MOLCAS' /
 &smu nprint=1
 orb='pentenal.ScfOrb' /

 &oao /
C* 1s(1) pr=1
C* 1s(2) pr=2
C* 2p(1) pr=2
O* 1s(1) pr=1
O* 1s(2) pr=2
O* 2p(1) pr=2
H* 1s(1) pr=2
fin

 &orb numac=57,60  /
C_C1 C1 1s(1) (1 0)
C_C2 C2 1s(1) (1 0)
C_C3 C3 1s(1) (1 0)
C_C4 C4 1s(1) (1 0)
C_C5 C5 1s(1) (1 0)
C_O1 O1 1s(1) (1 0)
O_O1 O1 1s(2) 2p(1) (2 0) Ref 1
A_O1C1 C1 1s(2) 2p(1) : O1 1s(2) 2p(1) (4 2) Proj 1
O_C1C2 C1 1s(2) 2p(1) : C2 1s(2) 2p(1) (1 1)
O_C2C3 C2 1s(2) 2p(1) : C3 1s(2) 2p(1) (2 2)
O_C3C4 C3 1s(2) 2p(1) : C4 1s(2) 2p(1) (1 1)
O_C4C5 C4 1s(2) 2p(1) : C5 1s(2) 2p(1) (2 2)
O_C1H1 C1 1s(2) 2p(1) : H1 1s(1) (1 1)
O_C2H2 C2 1s(2) 2p(1) : H2 1s(1) (1 1)
O_C3H3 C3 1s(2) 2p(1) : H3 1s(1) (1 1)
O_C4H4 C4 1s(2) 2p(1) : H4 1s(1) (1 1)
O_C5H51 C5 1s(2) 2p(1) : H51 1s(1) (1 1)
O_C5H52 C5 1s(2) 2p(1) : H52 1s(1) (1 1)
fin
\end{verbatim}

The oao section specifies the atomic orbital specification. For each atom you
need to specify the "priority" for the orbital lowdin orthogonalization. Please
note that:

1) the * matches numbers, not characters. This to clarify that C* does
\underline{not} match a Cr (chromium) atom.  This is against the bash common
behaviour for pattern globbing. Using C* matches every carbon atom. This
recall us the point that in seward the denomination must follow a consistent
rule.

2) For the specification of atomic orbitals, the 1s orbital is named 1s(1), the
2s is 1s(2) and so on.  The 2p is 2p(1), the 3p is 2p(2) and so on.

The \&orb section specifies the molecular orbitals that we want to build. The
first label of each line is made to recall the specification in seward. Also,
the first character of this label must be one of the follows

C for core orbitals
O for occupied
A for active
G for frozen

the frozen option is used to freeze the orbitals. no computation will be carried
over these orbitals and they will be taken as specified. Also, The subsequent
codes use components from molcas, and the specified orbitals will be tagged
with the Frozen molcas keyword. Frozing is useful when doing very large
molecules.

Core are orbitals that are deep in the orbital energy progression. They don't
produce virtual orbitals.

\begin{verbatim}
C_C1 C1 1s(1) (1 0)
\end{verbatim}

on this line, is specified that the 1s orbital for the C1 atom is core, and it
creates (1 0), 1 completely filled orbital and 0 virtual orbitals. the same for
other atoms.

Then it comes the lines

\begin{verbatim}
O_O1 O1 1s(2) 2p(1) (2 0) Ref 1
A_O1C1 C1 1s(2) 2p(1) : O1 1s(2) 2p(1) (4 2) Proj 1
\end{verbatim}

the first line tags an occupied group. There is a lot of stuff to keep in mind
here:

in the first line, the n type orbitals for the oxygen are tagged as occupied.
since we specified 2p(1), the px, py and pz are all kept to generate the
submatrix. For this reason, a 4x4 matrix will be extracted. Each column
represent the ny, nz, the half sigma and half pi orbitals that arise from the
combination, respectively. We say half because the bond is not specified, and
so in these orbitals goes only one electron. Then, we take out as occupied the
first two (which have a nearly 2.0 occupation) and tag as occupied.

Then it comes the A line. On that line we specify the C1 - O1 bond. we use 4
orbitals from the carbon and 4 orbitals from the oxygen, then a 8x8 matrix will
be extracted. In this matrix we found columns for the ny and nz oxygen
orbitals, the sigma CO bond, the sigma* CO bond, the pi CO bond, the pi* CO
bond, and finally two half bonds from the carbon to other atoms of the
molecule. We take out 4 nearly 2.0 occupated orbitals (ny, nz, pi, sigma) and 2
nearly 0.0 occupated orbitals (pi*, sigma*), and we drop the half bonds.

Please note that ny and nz are redundant, and also specified as Active, in this
case. Here comes two important notions: 1) the earlier specification of these
orbitals as occupied superimpose that indeed these orbitals \underline{are} occupied, not
active. 2) we can improve the cleanness of these n orbitals doing a projection
with the keywords Proj and Ref. The subsequent number is a tag, which says
"project the orbitals obtained from this line on the orbitals described at the
line tagged with Ref <same number>".

Finally, we can narrow again the choice of the active orbitals with the initial
numac keyword. Supposing we need the pi and pi* orbitals as active, we can do a
first run without numac, and then choose the appropriate number and rerun the
schmudorb picking out only the needed orbitals.

The other lines don't report differences. It's a simple creation of the
remaining bonds of the molecule.

Then it comes the program proj\_scf. the input is simple:

\begin{verbatim}
 &pscf prefix='pentenal.', /
\end{verbatim}

this projects the obtained guess over an scf calculation. This gives out
orbitals that are SCF in nature but localized. This improves the energy of our
orbitals, that is rather poor if not projected.  The code produces an output,
NONORLOC\_scf, that must be copied in NONORLOC

The program schmudort orthogonalizes the obtained orbitals. The input is rather
simple

\begin{verbatim}
 &ort
\end{verbatim}

and produces an output named LOCORB. This is copied to INPORB and fed to the
noscf chain, a sort of wrapper for various molcas programs. noscf calls motra,
molcost, casdi etc, to produce an improved (via variational) density matrix. At
Ferrara, we use a perturbative approach to create a perturbative density matrix
correction which is merged with the actual density matrix. This produces
improved orbitals, and the stop criteria is the convergence for the energy
value.

for noscf this is the input

\begin{verbatim}
 &nofil prefix='pentenal.'  /
 &no data='AUTO',r=' > ', MOLCAS=54,
 nmat=2,netat=2*1,metat=2*1,coef=2*1.d0 /
 &casdi syspin='+',gener='CAS+S',
 nelac=2,iprec=6,   /
 &casdi syspin='+',gener='CAS+S',
 nelac=2,iprec=6,is0=2   /
DAV1
 nvec=1
 / 
FIN
DAV2
 nvec=1
 /
FIN
\end{verbatim}


the first label is data='AUTO'. Another choice is MANUAL, but is used only to
test the chain.  r=' > ' command the writeout mode. with an > files with
incremental numbers for each step will be created. With \#, a single file with
all the steps will be written.
syspin is for the choice of the spin symmetry. When you specify is2=0 (which
is the default, for singlet) you also generate the zero component for the
triplet, quintet and so on. syspin = '+' match only the zero component for
the singlet.

then it comes the state specification keywords. nmat is the number of density
matrices we use.  netat is the number of states for each symmetry, metat is the
progressive number of the states we are interested to and coef are the
coefficients for each state.

Let's suppose we want to do a single state calculation, on the third state of
the first symmetry.  Then we need only one density matrix (nmat=1), the states
we want is one (netat=1), the state progressive number is 3 (metat=3), and the
weight is coef=1.0d0.

Suppose we want to do 4 states of the same symmetry. so we need nmat=1 (one
density matrix), netat=4 (four states) metat=1,2,3,4, and coef=4*1.0d0 (which
is equivalent to coef=1.0d0,1.0d0,1.0d0,1.0d0)

Suppose we want to do 2 states of different symmetry: for the first symmetry we
want the first state and for the fourth symmetry we want the third state. we
need to do nmat=2 (we need two density matrices since the symmetries of the
states is different) netat=1,1 (one of a symmetry, one of another symmetry,
please note that there's no evidence that the symmetries are the first and the
fourth, not yet), metat=1,4, coef=2*1.0d0

Then, subsequent \&casdi entries must be given for each of the states specified.
gener is the generator (CAS+S = CAS + singles) nelac = number of active
electrons. iprec= precision for the writeout of the results. is0 = the symmetry
we are interested to (in the precedent case, 1 or 4). ms2=the spin
multiplicity.

DAVn match the casdi entries.


Remember: check in the last molcost.out that the enumeration of orbitals has
changed or not. This must be checked since if you need to do subsequent
calculations the correct orbital number must be extracted for the active
space, and the given procedure can change the enumeration. Normally the
progression of the orbitals is made as: all G, all C, O occupied, A
occupied, A virtual, O virtual
Also: the target for localization is keeping the active space as little as
possible. Bear in mind that inserting in the active space orbitals that have
occupation number near 2 (or 0) a heavy coupling between these orbitals and
the core (or virtual) orbitals happens. For this reason, a lot of
convergence problems can arise. always keep the active space as little as
you can to describe the physic of your problem, and never enlarge it too
much.

For this reason, doing a ground state for the excitation energy evaluation
is better obtained with a single reference calculation (HF), and a CAS 2/2
for the excited state can be ok. A problem arise here: doing a CAS+S on the
excited state and comparing this energy with the HF energy can give negative
transition energy. This is a direct consequence that the excited state is
described better than the ground state. An averaged calculation must be done
to ensure a correct behaviour.

\section{Post optimization calculations}

After the optimization has taken place, the obtained orbitals can be found
in a file named LOCORB2, in the scratch file.
ok... now let's suppose we did an averaged treatment on ground and excited
states. We optimized the orbitals for both the states
These orbitals can be used to do subsequent evaluations at various level of
theory. Here we expose a CAS+SD treatment. The LOCORB2 file is taken and fed
to motra

cp \$CurrDir/LOCORB2 \$WorkDir/INPORB
molcas run motra   \$CurrDir/\$Project.motra.in >\$CurrDir/\$Project.motra.out

nothing dark magic here

 \&MOTRA \&END
Title
 acrolein
LumOrb
Frozen
 0 0 0 0
End of input

then we do a molcost

\&cost prefix='acrolein.',
 fermi=13,2,
/

then we run

casdet <  \$CurrDir/\$Project.casdi-ground.in > \$CurrDir/\$Project.casdi-ground.out
casdi <   \$CurrDir/\$Project.casdi-ground.in >> \$CurrDir/\$Project.casdi-ground.out

casdet generates the new space, with this input, and casdi computes the
energy value for the new wavefunction. This is the cumulated input

 \&cdfil prefix='acrolein.'  /
 \&cd gener='CAS+SD',noac=2,numac=13,39,
 nelac=2,is0=2,  /
 \&cdifil prefix='acrolein.'  /
 \&dav nvec=1,syspin='+' /

please \underline{note} that syspin goes into \&dav section, not \&cd section
in resemble to the input for noscf. This is an error that for singlet excited states
can produce the triplet, because the default for syspin is '0'!
Also, note that the enumeration of active orbitals can change, so check them
against the molcost output, taking out the numbers that are assigned to A\_
labelled orbitals.
casdi does not need the orbitals. It needs only the MO mono and bi
integrals. The AO integrals remain always the same (they are calculated at
the beginning, with seward) but for every calculation we use MO integrals,
which \underline{depend} on orbitals through the transformation achieved
with motra. Every post treatment does not need references to AO values, but
only mono and bi MO integrals (normally).

\end{document}
